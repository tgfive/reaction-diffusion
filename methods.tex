%% this is necessary stuff to make the document compile:
\documentclass[11pt]{amsart}
\usepackage{amssymb, amsmath}
\usepackage{colordvi,verbatim,hyperref}
\usepackage{color,enumitem}
\usepackage{graphicx}
\usepackage{amsthm}

%% this is for making margins the way I like them:
\headheight=8pt
\topmargin=0.375truein
\topmargin=-0.2truein
\textheight=9truein   \textwidth=6.3truein
\oddsidemargin=.1in \evensidemargin=.1in

%%this is to create new math commands. In other words, since I use theta^hat in equations a lot, I made a shorthand for it. So now instead of using $\hat\theta$ every time, I can just write $\thhat$. 
\newcommand{\Q}{{\mathbb{Q}}}
\newcommand{\N}{{\mathbb{N}}}
\newcommand{\R}{{\mathbb{R}}}

\pagestyle{plain}


\date{\today}%7 Feb 2014}

\begin{document}
\title{Analogy}
\author{Trent Gerew}

\maketitle

	The probability density $\rho (x,t)$ evolves according to the Fokker-Planck equation
	\begin{equation} \label{eq:f-p}
		\rho_t = \nabla \cdot \left( \rho \nabla \phi + \nabla \rho \right)
	\end{equation}
	where $\phi (x)$ is the potential energy.
\end{document}